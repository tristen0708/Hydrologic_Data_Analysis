\documentclass[]{article}
\usepackage{lmodern}
\usepackage{amssymb,amsmath}
\usepackage{ifxetex,ifluatex}
\usepackage{fixltx2e} % provides \textsubscript
\ifnum 0\ifxetex 1\fi\ifluatex 1\fi=0 % if pdftex
  \usepackage[T1]{fontenc}
  \usepackage[utf8]{inputenc}
\else % if luatex or xelatex
  \ifxetex
    \usepackage{mathspec}
  \else
    \usepackage{fontspec}
  \fi
  \defaultfontfeatures{Ligatures=TeX,Scale=MatchLowercase}
\fi
% use upquote if available, for straight quotes in verbatim environments
\IfFileExists{upquote.sty}{\usepackage{upquote}}{}
% use microtype if available
\IfFileExists{microtype.sty}{%
\usepackage{microtype}
\UseMicrotypeSet[protrusion]{basicmath} % disable protrusion for tt fonts
}{}
\usepackage[margin=2.54cm]{geometry}
\usepackage{hyperref}
\hypersetup{unicode=true,
            pdftitle={Assignment 5: Water Quality in Lakes},
            pdfauthor={Tristen Townsend},
            pdfborder={0 0 0},
            breaklinks=true}
\urlstyle{same}  % don't use monospace font for urls
\usepackage{color}
\usepackage{fancyvrb}
\newcommand{\VerbBar}{|}
\newcommand{\VERB}{\Verb[commandchars=\\\{\}]}
\DefineVerbatimEnvironment{Highlighting}{Verbatim}{commandchars=\\\{\}}
% Add ',fontsize=\small' for more characters per line
\usepackage{framed}
\definecolor{shadecolor}{RGB}{248,248,248}
\newenvironment{Shaded}{\begin{snugshade}}{\end{snugshade}}
\newcommand{\AlertTok}[1]{\textcolor[rgb]{0.94,0.16,0.16}{#1}}
\newcommand{\AnnotationTok}[1]{\textcolor[rgb]{0.56,0.35,0.01}{\textbf{\textit{#1}}}}
\newcommand{\AttributeTok}[1]{\textcolor[rgb]{0.77,0.63,0.00}{#1}}
\newcommand{\BaseNTok}[1]{\textcolor[rgb]{0.00,0.00,0.81}{#1}}
\newcommand{\BuiltInTok}[1]{#1}
\newcommand{\CharTok}[1]{\textcolor[rgb]{0.31,0.60,0.02}{#1}}
\newcommand{\CommentTok}[1]{\textcolor[rgb]{0.56,0.35,0.01}{\textit{#1}}}
\newcommand{\CommentVarTok}[1]{\textcolor[rgb]{0.56,0.35,0.01}{\textbf{\textit{#1}}}}
\newcommand{\ConstantTok}[1]{\textcolor[rgb]{0.00,0.00,0.00}{#1}}
\newcommand{\ControlFlowTok}[1]{\textcolor[rgb]{0.13,0.29,0.53}{\textbf{#1}}}
\newcommand{\DataTypeTok}[1]{\textcolor[rgb]{0.13,0.29,0.53}{#1}}
\newcommand{\DecValTok}[1]{\textcolor[rgb]{0.00,0.00,0.81}{#1}}
\newcommand{\DocumentationTok}[1]{\textcolor[rgb]{0.56,0.35,0.01}{\textbf{\textit{#1}}}}
\newcommand{\ErrorTok}[1]{\textcolor[rgb]{0.64,0.00,0.00}{\textbf{#1}}}
\newcommand{\ExtensionTok}[1]{#1}
\newcommand{\FloatTok}[1]{\textcolor[rgb]{0.00,0.00,0.81}{#1}}
\newcommand{\FunctionTok}[1]{\textcolor[rgb]{0.00,0.00,0.00}{#1}}
\newcommand{\ImportTok}[1]{#1}
\newcommand{\InformationTok}[1]{\textcolor[rgb]{0.56,0.35,0.01}{\textbf{\textit{#1}}}}
\newcommand{\KeywordTok}[1]{\textcolor[rgb]{0.13,0.29,0.53}{\textbf{#1}}}
\newcommand{\NormalTok}[1]{#1}
\newcommand{\OperatorTok}[1]{\textcolor[rgb]{0.81,0.36,0.00}{\textbf{#1}}}
\newcommand{\OtherTok}[1]{\textcolor[rgb]{0.56,0.35,0.01}{#1}}
\newcommand{\PreprocessorTok}[1]{\textcolor[rgb]{0.56,0.35,0.01}{\textit{#1}}}
\newcommand{\RegionMarkerTok}[1]{#1}
\newcommand{\SpecialCharTok}[1]{\textcolor[rgb]{0.00,0.00,0.00}{#1}}
\newcommand{\SpecialStringTok}[1]{\textcolor[rgb]{0.31,0.60,0.02}{#1}}
\newcommand{\StringTok}[1]{\textcolor[rgb]{0.31,0.60,0.02}{#1}}
\newcommand{\VariableTok}[1]{\textcolor[rgb]{0.00,0.00,0.00}{#1}}
\newcommand{\VerbatimStringTok}[1]{\textcolor[rgb]{0.31,0.60,0.02}{#1}}
\newcommand{\WarningTok}[1]{\textcolor[rgb]{0.56,0.35,0.01}{\textbf{\textit{#1}}}}
\usepackage{graphicx,grffile}
\makeatletter
\def\maxwidth{\ifdim\Gin@nat@width>\linewidth\linewidth\else\Gin@nat@width\fi}
\def\maxheight{\ifdim\Gin@nat@height>\textheight\textheight\else\Gin@nat@height\fi}
\makeatother
% Scale images if necessary, so that they will not overflow the page
% margins by default, and it is still possible to overwrite the defaults
% using explicit options in \includegraphics[width, height, ...]{}
\setkeys{Gin}{width=\maxwidth,height=\maxheight,keepaspectratio}
\IfFileExists{parskip.sty}{%
\usepackage{parskip}
}{% else
\setlength{\parindent}{0pt}
\setlength{\parskip}{6pt plus 2pt minus 1pt}
}
\setlength{\emergencystretch}{3em}  % prevent overfull lines
\providecommand{\tightlist}{%
  \setlength{\itemsep}{0pt}\setlength{\parskip}{0pt}}
\setcounter{secnumdepth}{0}
% Redefines (sub)paragraphs to behave more like sections
\ifx\paragraph\undefined\else
\let\oldparagraph\paragraph
\renewcommand{\paragraph}[1]{\oldparagraph{#1}\mbox{}}
\fi
\ifx\subparagraph\undefined\else
\let\oldsubparagraph\subparagraph
\renewcommand{\subparagraph}[1]{\oldsubparagraph{#1}\mbox{}}
\fi

%%% Use protect on footnotes to avoid problems with footnotes in titles
\let\rmarkdownfootnote\footnote%
\def\footnote{\protect\rmarkdownfootnote}

%%% Change title format to be more compact
\usepackage{titling}

% Create subtitle command for use in maketitle
\providecommand{\subtitle}[1]{
  \posttitle{
    \begin{center}\large#1\end{center}
    }
}

\setlength{\droptitle}{-2em}

  \title{Assignment 5: Water Quality in Lakes}
    \pretitle{\vspace{\droptitle}\centering\huge}
  \posttitle{\par}
    \author{Tristen Townsend}
    \preauthor{\centering\large\emph}
  \postauthor{\par}
    \date{}
    \predate{}\postdate{}
  

\begin{document}
\maketitle

\hypertarget{overview}{%
\subsection{OVERVIEW}\label{overview}}

This exercise accompanies the lessons in Hydrologic Data Analysis on
water quality in lakes

\hypertarget{directions}{%
\subsection{Directions}\label{directions}}

\begin{enumerate}
\def\labelenumi{\arabic{enumi}.}
\tightlist
\item
  Change ``Student Name'' on line 3 (above) with your name.
\item
  Work through the steps, \textbf{creating code and output} that fulfill
  each instruction.
\item
  Be sure to \textbf{answer the questions} in this assignment document.
\item
  When you have completed the assignment, \textbf{Knit} the text and
  code into a single HTML file.
\item
  After Knitting, submit the completed exercise (HTML file) to the
  dropbox in Sakai. Add your last name into the file name (e.g.,
  ``A05\_Salk.html'') prior to submission.
\end{enumerate}

The completed exercise is due on 2 October 2019 at 9:00 am.

\hypertarget{setup}{%
\subsection{Setup}\label{setup}}

\begin{enumerate}
\def\labelenumi{\arabic{enumi}.}
\tightlist
\item
  Verify your working directory is set to the R project file,
\item
  Load the tidyverse, lubridate, and LAGOSNE packages.
\item
  Set your ggplot theme (can be theme\_classic or something else)
\item
  Load the LAGOSdata database and the trophic state index csv file we
  created on 2019/09/27.
\end{enumerate}

\begin{Shaded}
\begin{Highlighting}[]
\KeywordTok{getwd}\NormalTok{()}
\end{Highlighting}
\end{Shaded}

\begin{verbatim}
## [1] "/Users/Tristen/OneDrive - Duke University/Fall 2019/Hydrologic Data Analysis/Hydrologic_Data_Analysis"
\end{verbatim}

\begin{Shaded}
\begin{Highlighting}[]
\KeywordTok{library}\NormalTok{(tidyverse)}
\end{Highlighting}
\end{Shaded}

\begin{verbatim}
## -- Attaching packages ------------------------------------- tidyverse 1.2.1 --
\end{verbatim}

\begin{verbatim}
## v ggplot2 3.2.1     v purrr   0.3.2
## v tibble  2.1.3     v dplyr   0.8.3
## v tidyr   0.8.3     v stringr 1.4.0
## v readr   1.3.1     v forcats 0.4.0
\end{verbatim}

\begin{verbatim}
## -- Conflicts ---------------------------------------- tidyverse_conflicts() --
## x dplyr::filter() masks stats::filter()
## x dplyr::lag()    masks stats::lag()
\end{verbatim}

\begin{Shaded}
\begin{Highlighting}[]
\KeywordTok{library}\NormalTok{(lubridate)}
\end{Highlighting}
\end{Shaded}

\begin{verbatim}
## 
## Attaching package: 'lubridate'
\end{verbatim}

\begin{verbatim}
## The following object is masked from 'package:base':
## 
##     date
\end{verbatim}

\begin{Shaded}
\begin{Highlighting}[]
\KeywordTok{library}\NormalTok{(LAGOSNE)}

\KeywordTok{theme_set}\NormalTok{(}\KeywordTok{theme_classic}\NormalTok{())}
\KeywordTok{options}\NormalTok{(}\DataTypeTok{scipen =} \DecValTok{100}\NormalTok{)}

\KeywordTok{load}\NormalTok{(}\DataTypeTok{file =} \StringTok{"./Data/Raw/LAGOSdata.rda"}\NormalTok{)}
\NormalTok{LAGOStrophic <-}\StringTok{ }\KeywordTok{read_csv}\NormalTok{(}\DataTypeTok{file =} \StringTok{"./Data/LAGOStrophic.csv"}\NormalTok{)}
\end{Highlighting}
\end{Shaded}

\begin{verbatim}
## Parsed with column specification:
## cols(
##   lagoslakeid = col_double(),
##   sampledate = col_date(format = ""),
##   chla = col_double(),
##   tp = col_double(),
##   secchi = col_double(),
##   gnis_name = col_character(),
##   lake_area_ha = col_double(),
##   state = col_character(),
##   state_name = col_character(),
##   sampleyear = col_double(),
##   samplemonth = col_double(),
##   season = col_character(),
##   TSI.chl = col_double(),
##   TSI.secchi = col_double(),
##   TSI.tp = col_double(),
##   trophic.class = col_character()
## )
\end{verbatim}

\hypertarget{trophic-state-index}{%
\subsection{Trophic State Index}\label{trophic-state-index}}

\begin{enumerate}
\def\labelenumi{\arabic{enumi}.}
\setcounter{enumi}{4}
\tightlist
\item
  Similar to the trophic.class column we created in class (determined
  from TSI.chl values), create two additional columns in the data frame
  that determine trophic class from TSI.secchi and TSI.tp (call these
  trophic.class.secchi and trophic.class.tp).
\end{enumerate}

\begin{Shaded}
\begin{Highlighting}[]
\NormalTok{LAGOStrophic <-}\StringTok{ }
\StringTok{  }\KeywordTok{mutate}\NormalTok{(LAGOStrophic, }
         \DataTypeTok{TSI.chl =} \KeywordTok{round}\NormalTok{(}\DecValTok{10}\OperatorTok{*}\NormalTok{(}\DecValTok{6} \OperatorTok{-}\StringTok{ }\NormalTok{(}\FloatTok{2.04} \OperatorTok{-}\StringTok{ }\FloatTok{0.68}\OperatorTok{*}\KeywordTok{log}\NormalTok{(chla)}\OperatorTok{/}\KeywordTok{log}\NormalTok{(}\DecValTok{2}\NormalTok{)))),}
         \DataTypeTok{TSI.secchi =} \KeywordTok{round}\NormalTok{(}\DecValTok{10}\OperatorTok{*}\NormalTok{(}\DecValTok{6} \OperatorTok{-}\StringTok{ }\NormalTok{(}\KeywordTok{log}\NormalTok{(secchi)}\OperatorTok{/}\KeywordTok{log}\NormalTok{(}\DecValTok{2}\NormalTok{)))), }
         \DataTypeTok{TSI.tp =} \KeywordTok{round}\NormalTok{(}\DecValTok{10}\OperatorTok{*}\NormalTok{(}\DecValTok{6} \OperatorTok{-}\StringTok{ }\NormalTok{(}\KeywordTok{log}\NormalTok{(}\DecValTok{48}\OperatorTok{/}\NormalTok{tp)}\OperatorTok{/}\KeywordTok{log}\NormalTok{(}\DecValTok{2}\NormalTok{)))), }
         \DataTypeTok{trophic.class.secchi =} 
            \KeywordTok{ifelse}\NormalTok{(TSI.secchi }\OperatorTok{<}\StringTok{ }\DecValTok{40}\NormalTok{, }\StringTok{"Oligotrophic"}\NormalTok{, }
                   \KeywordTok{ifelse}\NormalTok{(TSI.secchi }\OperatorTok{<}\StringTok{ }\DecValTok{50}\NormalTok{, }\StringTok{"Mesotrophic"}\NormalTok{,}
                          \KeywordTok{ifelse}\NormalTok{(TSI.secchi }\OperatorTok{<}\StringTok{ }\DecValTok{70}\NormalTok{, }\StringTok{"Eutrophic"}\NormalTok{, }\StringTok{"Hypereutrophic"}\NormalTok{))),}
         \DataTypeTok{trophic.class.tp =} 
           \KeywordTok{ifelse}\NormalTok{(TSI.tp }\OperatorTok{<}\StringTok{ }\DecValTok{40}\NormalTok{, }\StringTok{"Oligotrophic"}\NormalTok{,}
                  \KeywordTok{ifelse}\NormalTok{(TSI.tp }\OperatorTok{<}\StringTok{ }\DecValTok{50}\NormalTok{, }\StringTok{"Mesotrophic"}\NormalTok{,}
                         \KeywordTok{ifelse}\NormalTok{(TSI.tp }\OperatorTok{<}\StringTok{ }\DecValTok{70}\NormalTok{, }\StringTok{"Eutrophic"}\NormalTok{, }\StringTok{"Hypereutrophic"}\NormalTok{))))}

\NormalTok{LAGOStrophic}\OperatorTok{$}\NormalTok{trophic.class <-}\StringTok{ }
\StringTok{  }\KeywordTok{factor}\NormalTok{(LAGOStrophic}\OperatorTok{$}\NormalTok{trophic.class,}
         \DataTypeTok{levels =} \KeywordTok{c}\NormalTok{(}\StringTok{"Oligotrophic"}\NormalTok{, }\StringTok{"Mesotrophic"}\NormalTok{, }\StringTok{"Eutrophic"}\NormalTok{, }\StringTok{"Hypereutrophic"}\NormalTok{))}

\NormalTok{LAGOStrophic}\OperatorTok{$}\NormalTok{trophic.class.secchi <-}\StringTok{ }
\StringTok{  }\KeywordTok{factor}\NormalTok{(LAGOStrophic}\OperatorTok{$}\NormalTok{trophic.class.secchi,}
         \DataTypeTok{levels =} \KeywordTok{c}\NormalTok{(}\StringTok{"Oligotrophic"}\NormalTok{, }\StringTok{"Mesotrophic"}\NormalTok{, }\StringTok{"Eutrophic"}\NormalTok{, }\StringTok{"Hypereutrophic"}\NormalTok{))}

\NormalTok{LAGOStrophic}\OperatorTok{$}\NormalTok{trophic.class.tp <-}\StringTok{ }
\StringTok{  }\KeywordTok{factor}\NormalTok{(LAGOStrophic}\OperatorTok{$}\NormalTok{trophic.class.tp,}
         \DataTypeTok{levels =} \KeywordTok{c}\NormalTok{(}\StringTok{"Oligotrophic"}\NormalTok{, }\StringTok{"Mesotrophic"}\NormalTok{, }\StringTok{"Eutrophic"}\NormalTok{, }\StringTok{"Hypereutrophic"}\NormalTok{))}
\end{Highlighting}
\end{Shaded}

\begin{enumerate}
\def\labelenumi{\arabic{enumi}.}
\setcounter{enumi}{5}
\tightlist
\item
  How many observations fall into the four trophic state categories for
  the three metrics (trophic.class, trophic.class.secchi,
  trophic.class.tp)? Hint: \texttt{count} function.
\end{enumerate}

\begin{Shaded}
\begin{Highlighting}[]
\NormalTok{count.class <-}\StringTok{ }\KeywordTok{count}\NormalTok{(LAGOStrophic, trophic.class)}
\NormalTok{count.secchi <-}\StringTok{ }\KeywordTok{count}\NormalTok{(LAGOStrophic, trophic.class.secchi)}
\NormalTok{count.tp <-}\StringTok{ }\KeywordTok{count}\NormalTok{(LAGOStrophic, trophic.class.tp)}
\end{Highlighting}
\end{Shaded}

\begin{enumerate}
\def\labelenumi{\arabic{enumi}.}
\setcounter{enumi}{6}
\tightlist
\item
  What proportion of total observations are considered eutrohic or
  hypereutrophic according to the three different metrics
  (trophic.class, trophic.class.secchi, trophic.class.tp)?
\end{enumerate}

\begin{Shaded}
\begin{Highlighting}[]
\CommentTok{#Trophic class - Chl.A}
\CommentTok{#Eutrophic}
\NormalTok{count.class[}\DecValTok{3}\NormalTok{,}\DecValTok{2}\NormalTok{]}\OperatorTok{/}\KeywordTok{sum}\NormalTok{(count.class}\OperatorTok{$}\NormalTok{n) }\CommentTok{#0.559}
\end{Highlighting}
\end{Shaded}

\begin{verbatim}
##           n
## 1 0.5585116
\end{verbatim}

\begin{Shaded}
\begin{Highlighting}[]
\CommentTok{#Hypereutrophic}
\NormalTok{count.class[}\DecValTok{4}\NormalTok{,}\DecValTok{2}\NormalTok{]}\OperatorTok{/}\KeywordTok{sum}\NormalTok{(count.class}\OperatorTok{$}\NormalTok{n) }\CommentTok{#0.192}
\end{Highlighting}
\end{Shaded}

\begin{verbatim}
##           n
## 1 0.1918453
\end{verbatim}

\begin{Shaded}
\begin{Highlighting}[]
\CommentTok{#Trophic class - Secchi}
\CommentTok{#Eutrophic}
\NormalTok{count.secchi[}\DecValTok{3}\NormalTok{,}\DecValTok{2}\NormalTok{]}\OperatorTok{/}\KeywordTok{sum}\NormalTok{(count.secchi}\OperatorTok{$}\NormalTok{n) }\CommentTok{#0.382}
\end{Highlighting}
\end{Shaded}

\begin{verbatim}
##           n
## 1 0.3823698
\end{verbatim}

\begin{Shaded}
\begin{Highlighting}[]
\CommentTok{#Hypereutrophic}
\NormalTok{count.secchi[}\DecValTok{4}\NormalTok{,}\DecValTok{2}\NormalTok{]}\OperatorTok{/}\KeywordTok{sum}\NormalTok{(count.secchi}\OperatorTok{$}\NormalTok{n) }\CommentTok{#0.068}
\end{Highlighting}
\end{Shaded}

\begin{verbatim}
##            n
## 1 0.06803111
\end{verbatim}

\begin{Shaded}
\begin{Highlighting}[]
\CommentTok{#Trophic class - TP}
\CommentTok{#Eutrophic}
\NormalTok{count.tp[}\DecValTok{3}\NormalTok{,}\DecValTok{2}\NormalTok{]}\OperatorTok{/}\KeywordTok{sum}\NormalTok{(count.tp}\OperatorTok{$}\NormalTok{n) }\CommentTok{#0.331}
\end{Highlighting}
\end{Shaded}

\begin{verbatim}
##           n
## 1 0.3314032
\end{verbatim}

\begin{Shaded}
\begin{Highlighting}[]
\CommentTok{#Hypereutrophic}
\NormalTok{count.tp[}\DecValTok{4}\NormalTok{,}\DecValTok{2}\NormalTok{]}\OperatorTok{/}\KeywordTok{sum}\NormalTok{(count.tp}\OperatorTok{$}\NormalTok{n) }\CommentTok{#0.096}
\end{Highlighting}
\end{Shaded}

\begin{verbatim}
##            n
## 1 0.09643634
\end{verbatim}

Which of these metrics is most conservative in its designation of
eutrophic conditions? Why might this be?

\begin{quote}
Total phosphorus. This is likely because there are more things that
could influence the other two variables. For secchi depth, it can be
affected by things such as dissolved organics or sediments in the water.
And chorophyll-a has more things that can influence it's values that do
not affect total phosphorus concentrations (for example, nitrogen would
might increase chlorophyll-a but not total phosphorus).
\end{quote}

Note: To take this further, a researcher might determine which trophic
classes are susceptible to being differently categorized by the
different metrics and whether certain metrics are prone to categorizing
trophic class as more or less eutrophic. This would entail more complex
code.

\hypertarget{nutrient-concentrations}{%
\subsection{Nutrient Concentrations}\label{nutrient-concentrations}}

\begin{enumerate}
\def\labelenumi{\arabic{enumi}.}
\setcounter{enumi}{7}
\tightlist
\item
  Create a data frame that includes the columns lagoslakeid, sampledate,
  tn, tp, state, and state\_name. Mutate this data frame to include
  sampleyear and samplemonth columns as well. Call this data frame
  LAGOSNandP.
\end{enumerate}

\begin{Shaded}
\begin{Highlighting}[]
\NormalTok{LAGOSlocus <-}\StringTok{ }\NormalTok{LAGOSdata}\OperatorTok{$}\NormalTok{locus}
\NormalTok{LAGOSstate <-}\StringTok{ }\NormalTok{LAGOSdata}\OperatorTok{$}\NormalTok{state}
\NormalTok{LAGOSnutrient <-}\StringTok{ }\NormalTok{LAGOSdata}\OperatorTok{$}\NormalTok{epi_nutr}

\NormalTok{LAGOSlocus}\OperatorTok{$}\NormalTok{lagoslakeid <-}\StringTok{ }\KeywordTok{as.factor}\NormalTok{(LAGOSlocus}\OperatorTok{$}\NormalTok{lagoslakeid)}
\NormalTok{LAGOSnutrient}\OperatorTok{$}\NormalTok{lagoslakeid <-}\StringTok{ }\KeywordTok{as.factor}\NormalTok{(LAGOSnutrient}\OperatorTok{$}\NormalTok{lagoslakeid)}

\NormalTok{LAGOSlocations <-}\StringTok{ }\KeywordTok{left_join}\NormalTok{(LAGOSlocus, LAGOSstate, }\DataTypeTok{by =} \StringTok{"state_zoneid"}\NormalTok{)}

\NormalTok{LAGOSNandP <-}\StringTok{ }
\StringTok{  }\KeywordTok{left_join}\NormalTok{(LAGOSnutrient, LAGOSlocations, }\DataTypeTok{by =} \StringTok{"lagoslakeid"}\NormalTok{) }\OperatorTok
\StringTok{  }\KeywordTok{select}\NormalTok{(lagoslakeid, sampledate, tn, tp, state, state_name) }\OperatorTok
\StringTok{  }\KeywordTok{mutate}\NormalTok{(}\DataTypeTok{sampleyear =} \KeywordTok{year}\NormalTok{(sampledate), }
         \DataTypeTok{samplemonth =} \KeywordTok{month}\NormalTok{(sampledate))}
\end{Highlighting}
\end{Shaded}

\begin{verbatim}
## Warning: Column `lagoslakeid` joining factors with different levels,
## coercing to character vector
\end{verbatim}

\begin{enumerate}
\def\labelenumi{\arabic{enumi}.}
\setcounter{enumi}{8}
\tightlist
\item
  Create two violin plots comparing TN and TP concentrations across
  states. Include a 50th percentile line inside the violins.
\end{enumerate}

\begin{Shaded}
\begin{Highlighting}[]
\NormalTok{stateTNviolin <-}\StringTok{ }\KeywordTok{ggplot}\NormalTok{(LAGOSNandP, }\KeywordTok{aes}\NormalTok{(}\DataTypeTok{x =}\NormalTok{ state, }\DataTypeTok{y =}\NormalTok{ tn)) }\OperatorTok{+}
\StringTok{  }\KeywordTok{geom_violin}\NormalTok{(}\DataTypeTok{draw_quantiles =} \FloatTok{0.50}\NormalTok{)}
\KeywordTok{print}\NormalTok{(stateTNviolin)}
\end{Highlighting}
\end{Shaded}

\begin{verbatim}
## Warning: Removed 774226 rows containing non-finite values (stat_ydensity).
\end{verbatim}

\begin{verbatim}
## Warning in regularize.values(x, y, ties, missing(ties)): collapsing to
## unique 'x' values

## Warning in regularize.values(x, y, ties, missing(ties)): collapsing to
## unique 'x' values

## Warning in regularize.values(x, y, ties, missing(ties)): collapsing to
## unique 'x' values

## Warning in regularize.values(x, y, ties, missing(ties)): collapsing to
## unique 'x' values

## Warning in regularize.values(x, y, ties, missing(ties)): collapsing to
## unique 'x' values

## Warning in regularize.values(x, y, ties, missing(ties)): collapsing to
## unique 'x' values
\end{verbatim}

\includegraphics{A05_LakeWQ_files/figure-latex/unnamed-chunk-5-1.pdf}

\begin{Shaded}
\begin{Highlighting}[]
\NormalTok{stateTPviolin <-}\StringTok{ }\KeywordTok{ggplot}\NormalTok{(LAGOSNandP, }\KeywordTok{aes}\NormalTok{(}\DataTypeTok{x =}\NormalTok{ state, }\DataTypeTok{y =}\NormalTok{ tp)) }\OperatorTok{+}
\StringTok{  }\KeywordTok{geom_violin}\NormalTok{(}\DataTypeTok{draw_quantiles =} \FloatTok{0.50}\NormalTok{)}
\KeywordTok{print}\NormalTok{(stateTPviolin)}
\end{Highlighting}
\end{Shaded}

\begin{verbatim}
## Warning: Removed 672861 rows containing non-finite values (stat_ydensity).

## Warning: collapsing to unique 'x' values

## Warning: collapsing to unique 'x' values

## Warning: collapsing to unique 'x' values

## Warning: collapsing to unique 'x' values

## Warning: collapsing to unique 'x' values

## Warning: collapsing to unique 'x' values

## Warning: collapsing to unique 'x' values

## Warning: collapsing to unique 'x' values

## Warning: collapsing to unique 'x' values

## Warning: collapsing to unique 'x' values

## Warning: collapsing to unique 'x' values

## Warning: collapsing to unique 'x' values

## Warning: collapsing to unique 'x' values
\end{verbatim}

\includegraphics{A05_LakeWQ_files/figure-latex/unnamed-chunk-5-2.pdf}

Which states have the highest and lowest median concentrations?

\begin{quote}
TN: Highest: Iowa; Lowest: Vermont, New Hamsphire, and Maine (this
visualization makes it difficult to be certain)
\end{quote}

\begin{quote}
TP: Highest: Illinois; Lowest: Maine (though this visualization makes it
difficult to be certain)
\end{quote}

Which states have the highest and lowest concentration ranges?

\begin{quote}
TN: Highest range: Iowa; Lowest range: Vermont or New Hampshire
\end{quote}

\begin{quote}
TP: Highest range: Illinois, Massachusetts, Minnesota; Lowest range:
Pennsylvania
\end{quote}

\begin{enumerate}
\def\labelenumi{\arabic{enumi}.}
\setcounter{enumi}{9}
\tightlist
\item
  Create two jitter plots comparing TN and TP concentrations across
  states, with samplemonth as the color. Choose a color palette other
  than the ggplot default.
\end{enumerate}

\begin{Shaded}
\begin{Highlighting}[]
\NormalTok{LAGOSNandP}\OperatorTok{$}\NormalTok{state <-}\StringTok{ }\KeywordTok{as.factor}\NormalTok{(LAGOSNandP}\OperatorTok{$}\NormalTok{state)}
\NormalTok{LAGOSNandP}\OperatorTok{$}\NormalTok{state_name <-}\StringTok{ }\KeywordTok{as.factor}\NormalTok{(LAGOSNandP}\OperatorTok{$}\NormalTok{state_name)}

\NormalTok{stateTNjitter <-}\StringTok{ }\KeywordTok{ggplot}\NormalTok{(LAGOSNandP, }\KeywordTok{aes}\NormalTok{(}\DataTypeTok{x =}\NormalTok{ state_name, }\DataTypeTok{y =}\NormalTok{ tn, }\DataTypeTok{color =}\NormalTok{ samplemonth)) }\OperatorTok{+}
\StringTok{  }\KeywordTok{geom_jitter}\NormalTok{(}\DataTypeTok{alpha =} \FloatTok{0.3}\NormalTok{) }\OperatorTok{+}
\StringTok{  }\KeywordTok{labs}\NormalTok{(}\DataTypeTok{x =} \StringTok{""}\NormalTok{, }\DataTypeTok{y =} \StringTok{"TSI(tn)"}\NormalTok{) }\OperatorTok{+}
\StringTok{  }\KeywordTok{theme}\NormalTok{(}\DataTypeTok{legend.position =} \StringTok{"top"}\NormalTok{) }\OperatorTok{+}
\StringTok{  }\KeywordTok{scale_color_viridis_c}\NormalTok{(}\DataTypeTok{option =} \StringTok{"magma"}\NormalTok{) }\OperatorTok{+}
\StringTok{  }\KeywordTok{theme}\NormalTok{(}\DataTypeTok{axis.text.x =} \KeywordTok{element_text}\NormalTok{(}\DataTypeTok{angle =} \DecValTok{45}\NormalTok{, }\DataTypeTok{vjust =} \DecValTok{1}\NormalTok{, }\DataTypeTok{hjust=}\DecValTok{1}\NormalTok{))}
\KeywordTok{print}\NormalTok{(stateTNjitter)}
\end{Highlighting}
\end{Shaded}

\begin{verbatim}
## Warning: Removed 774226 rows containing missing values (geom_point).
\end{verbatim}

\includegraphics{A05_LakeWQ_files/figure-latex/unnamed-chunk-6-1.pdf}

\begin{Shaded}
\begin{Highlighting}[]
\NormalTok{stateTPjitter <-}\StringTok{ }\KeywordTok{ggplot}\NormalTok{(LAGOSNandP, }\KeywordTok{aes}\NormalTok{(}\DataTypeTok{x =}\NormalTok{ state_name, }\DataTypeTok{y =}\NormalTok{ tp, }\DataTypeTok{color =}\NormalTok{ samplemonth)) }\OperatorTok{+}
\StringTok{ }\KeywordTok{geom_jitter}\NormalTok{(}\DataTypeTok{alpha =} \FloatTok{0.3}\NormalTok{) }\OperatorTok{+}
\StringTok{  }\KeywordTok{labs}\NormalTok{(}\DataTypeTok{x =} \StringTok{""}\NormalTok{, }\DataTypeTok{y =} \StringTok{"TSI(tp)"}\NormalTok{) }\OperatorTok{+}
\StringTok{  }\KeywordTok{theme}\NormalTok{(}\DataTypeTok{legend.position =} \StringTok{"top"}\NormalTok{) }\OperatorTok{+}
\StringTok{  }\KeywordTok{scale_color_viridis_c}\NormalTok{(}\DataTypeTok{option =} \StringTok{"magma"}\NormalTok{)}\OperatorTok{+}
\StringTok{  }\KeywordTok{theme}\NormalTok{(}\DataTypeTok{axis.text.x =} \KeywordTok{element_text}\NormalTok{(}\DataTypeTok{angle =} \DecValTok{45}\NormalTok{, }\DataTypeTok{vjust =} \DecValTok{1}\NormalTok{, }\DataTypeTok{hjust=}\DecValTok{1}\NormalTok{))}
\KeywordTok{print}\NormalTok{(stateTPjitter)}
\end{Highlighting}
\end{Shaded}

\begin{verbatim}
## Warning: Removed 672861 rows containing missing values (geom_point).
\end{verbatim}

\includegraphics{A05_LakeWQ_files/figure-latex/unnamed-chunk-6-2.pdf}

Which states have the most samples? How might this have impacted total
ranges from \#9?

\begin{quote}
TN: Most: Iowa; Least: Vermont.
\end{quote}

\begin{quote}
TP: Most: Illinois, Minnesota, Wisconsin; Least: New Hampshire
\end{quote}

Which months are sampled most extensively? Does this differ among
states?

\begin{quote}
TN: It appears that the late summer months are the most extensively
samples. There does seem to be some differing among states, as some seem
to sample exclusively during the summer and some do so across multiple
seasons.
\end{quote}

\begin{quote}
TP: Again, it appears that the late summer months are the most
extensively samples. There does seem to be some differing among states,
as some seem to sample exclusively during the summer and some do so
across multiple seasons.
\end{quote}

\begin{enumerate}
\def\labelenumi{\arabic{enumi}.}
\setcounter{enumi}{10}
\tightlist
\item
  Create two jitter plots comparing TN and TP concentrations across
  states, with sampleyear as the color. Choose a color palette other
  than the ggplot default.
\end{enumerate}

\begin{Shaded}
\begin{Highlighting}[]
\NormalTok{stateTNjitter.year <-}\StringTok{ }\KeywordTok{ggplot}\NormalTok{(LAGOSNandP, }\KeywordTok{aes}\NormalTok{(}\DataTypeTok{x =}\NormalTok{ state_name, }\DataTypeTok{y =}\NormalTok{ tn, }\DataTypeTok{color =}\NormalTok{ sampleyear)) }\OperatorTok{+}
\StringTok{  }\KeywordTok{geom_jitter}\NormalTok{(}\DataTypeTok{alpha =} \FloatTok{0.3}\NormalTok{) }\OperatorTok{+}
\StringTok{  }\KeywordTok{labs}\NormalTok{(}\DataTypeTok{x =} \StringTok{""}\NormalTok{, }\DataTypeTok{y =} \StringTok{"TSI(tn)"}\NormalTok{) }\OperatorTok{+}
\StringTok{  }\KeywordTok{theme}\NormalTok{(}\DataTypeTok{legend.position =} \StringTok{"top"}\NormalTok{) }\OperatorTok{+}
\StringTok{  }\KeywordTok{scale_color_viridis_c}\NormalTok{(}\DataTypeTok{option =} \StringTok{"magma"}\NormalTok{) }\OperatorTok{+}
\StringTok{  }\KeywordTok{theme}\NormalTok{(}\DataTypeTok{axis.text.x =} \KeywordTok{element_text}\NormalTok{(}\DataTypeTok{angle =} \DecValTok{45}\NormalTok{, }\DataTypeTok{vjust =} \DecValTok{1}\NormalTok{, }\DataTypeTok{hjust=}\DecValTok{1}\NormalTok{))}
\KeywordTok{print}\NormalTok{(stateTNjitter.year)}
\end{Highlighting}
\end{Shaded}

\begin{verbatim}
## Warning: Removed 774226 rows containing missing values (geom_point).
\end{verbatim}

\includegraphics{A05_LakeWQ_files/figure-latex/unnamed-chunk-7-1.pdf}

\begin{Shaded}
\begin{Highlighting}[]
\NormalTok{stateTPjitter.year <-}\StringTok{ }\KeywordTok{ggplot}\NormalTok{(LAGOSNandP, }\KeywordTok{aes}\NormalTok{(}\DataTypeTok{x =}\NormalTok{ state_name, }\DataTypeTok{y =}\NormalTok{ tp, }\DataTypeTok{color =}\NormalTok{ sampleyear)) }\OperatorTok{+}
\StringTok{ }\KeywordTok{geom_jitter}\NormalTok{(}\DataTypeTok{alpha =} \FloatTok{0.3}\NormalTok{) }\OperatorTok{+}
\StringTok{  }\KeywordTok{labs}\NormalTok{(}\DataTypeTok{x =} \StringTok{""}\NormalTok{, }\DataTypeTok{y =} \StringTok{"TSI(tp)"}\NormalTok{) }\OperatorTok{+}
\StringTok{  }\KeywordTok{theme}\NormalTok{(}\DataTypeTok{legend.position =} \StringTok{"top"}\NormalTok{) }\OperatorTok{+}
\StringTok{  }\KeywordTok{scale_color_viridis_c}\NormalTok{(}\DataTypeTok{option =} \StringTok{"magma"}\NormalTok{)}\OperatorTok{+}
\StringTok{  }\KeywordTok{theme}\NormalTok{(}\DataTypeTok{axis.text.x =} \KeywordTok{element_text}\NormalTok{(}\DataTypeTok{angle =} \DecValTok{45}\NormalTok{, }\DataTypeTok{vjust =} \DecValTok{1}\NormalTok{, }\DataTypeTok{hjust=}\DecValTok{1}\NormalTok{))}
\KeywordTok{print}\NormalTok{(stateTPjitter.year)}
\end{Highlighting}
\end{Shaded}

\begin{verbatim}
## Warning: Removed 672861 rows containing missing values (geom_point).
\end{verbatim}

\includegraphics{A05_LakeWQ_files/figure-latex/unnamed-chunk-7-2.pdf}

Which years are sampled most extensively? Does this differ among states?

\begin{quote}
TN: It appears that the majority of samples are from 2000 to the
present. It doesn't seem to differ drastically between states.
\end{quote}

\begin{quote}
TP: It appears that most of the samples are from 2000 to the present,
but there's a decent amount that seem to date back to the mid-70s. The
TP samples have more more variation between states in regards to when
samples were collected.
\end{quote}

\hypertarget{reflection}{%
\subsection{Reflection}\label{reflection}}

\begin{enumerate}
\def\labelenumi{\arabic{enumi}.}
\setcounter{enumi}{11}
\tightlist
\item
  What are 2-3 conclusions or summary points about lake water quality
  you learned through your analysis?
\end{enumerate}

\begin{quote}
Different variables can indicate very different levels of
eutrophication. The time of year data is sampled can play a large role
on the trends seen in data.
\end{quote}

\begin{enumerate}
\def\labelenumi{\arabic{enumi}.}
\setcounter{enumi}{12}
\tightlist
\item
  What data, visualizations, and/or models supported your conclusions
  from 12?
\end{enumerate}

\begin{quote}
The trophic data where we looked at the proportion of total observations
which are considered eutrophic or hypereutrophic. (Count data
statistics) States that sampled more frequently and sampled across
seasons have larger ranges in their concentration data. (Jitter plots
showed us this)
\end{quote}

\begin{enumerate}
\def\labelenumi{\arabic{enumi}.}
\setcounter{enumi}{13}
\tightlist
\item
  Did hands-on data analysis impact your learning about water quality
  relative to a theory-based lesson? If so, how?
\end{enumerate}

\begin{quote}
Yes, having the opportunity to analyze real-world data allows for more
critical thinking skills to be used to understnad trends and apply
theoretical expectations to real-world data.
\end{quote}

\begin{enumerate}
\def\labelenumi{\arabic{enumi}.}
\setcounter{enumi}{14}
\tightlist
\item
  How did the real-world data compare with your expectations from
  theory?
\end{enumerate}

\begin{quote}
There real-world data illustrates how messy things can be, and how
sample size and timing (sampling decisions are made by humans!) can play
a huge factor in the types of trends you see.
\end{quote}


\end{document}
